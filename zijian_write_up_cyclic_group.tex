\documentclass[12pt]{article}
\usepackage{amsmath, amsthm}
\usepackage{amsfonts}
\usepackage{amssymb,amscd,epsf,latexsym,verbatim,graphicx}
\usepackage{mathrsfs}
\usepackage{graphicx}
\usepackage{latexsym}
\usepackage{lscape}
\usepackage[hidelinks]{hyperref}
\usepackage{epstopdf}
\usepackage{tikz}
\usetikzlibrary{calc}
\usetikzlibrary{matrix,arrows,decorations.pathmorphing}
\usepackage{tikz-cd}
\usepackage{color}
\usepackage{geometry}
\usepackage{multirow}
\usepackage[vcentermath]{youngtab}


%% To define Sha

\DeclareFontFamily{U}{wncy}{}
\DeclareFontShape{U}{wncy}{m}{n}{<->wncyr10}{}
\DeclareSymbolFont{mcy}{U}{wncy}{m}{n}
\DeclareMathSymbol{\Sh}{\mathord}{mcy}{"58} 

\linespread{1.2}
\renewcommand{\vec}[1]{\overrightarrow{#1}}
\newcommand{\cl}{\overline}
\newcommand{\n}{\vartriangleleft} %normal subgroup
\newcommand{\dstyle}{\displaystyle}
\newcommand{\Z}{\mathbb{Z}}
\newcommand{\1}{\mathbb{I}}
\newcommand{\N}{\mathbb{N}}
\newcommand{\R}{\mathbb{R}}
\newcommand{\Q}{\mathbb{Q}}
\newcommand{\C}{\mathbb{C}}
\renewcommand{\a}{\mathfrak{a}}
\renewcommand{\b}{\mathfrak{b}}
\newcommand{\p}{\mathfrak{p}}
\newcommand{\q}{\mathfrak{q}}
\newcommand{\m}{\mathfrak{m}}
\newcommand{\M}{\mathcal{M}}
\newcommand{\so}{\Rightarrow}
\newcommand{\upo}{\mathring}
\newcommand{\ra}{\rightarrow}
\newcommand{\la}{\leftarrow}
\newcommand{\sq}{\widetilde}
\renewcommand{\iff}{\Leftrightarrow}
%\newcommand{\im}{\text{im}}
\newcommand{\minus}{\backslash}
\renewcommand{\ss}{\text{sylow-subgroup}}
\renewcommand{\vec}[1]{\overrightarrow{#1}}
\renewcommand{\v}{\mathbf}
\newcommand{\limit}{\dstyle \lim_{n \ra \infty}}
\newcommand{\gr}[1]{\langle {#1} \rangle}
\newcommand{\nil}{\mathfrak{N}}

\usepackage{array}
\newcolumntype{C}[1]{>{\centering\let\newline\\\arraybackslash\hspace{0pt}}m{#1}}
\usepackage{multirow}
\usepackage{hhline}

\setcounter{section}{0} 
\theoremstyle{definition}
\newtheorem{theorem}{Theorem}[section]
\newtheorem{lemma}[theorem]{Lemma}
\newtheorem{proposition}[theorem]{Proposition}
\newtheorem{conjecture}[theorem]{Conjecture}
\newtheorem{corollary}[theorem]{Corollary}
\newtheorem{claim}[theorem]{Claim}
\theoremstyle{definition}
\newtheorem{remark}[theorem]{Remark}
\theoremstyle{definition}
\newtheorem{example}[theorem]{Example}
\theoremstyle{definition}
\newtheorem{definition}[theorem]{Definition}
\theoremstyle{definition}
\newtheorem{definitions}[theorem]{Definitions}
\theoremstyle{definition}
\newtheorem{outline}[theorem]{Outline of Proof}
\DeclareMathOperator{\coker}{coker}
\DeclareMathOperator{\im}{im}


%\intextsep 11pt plus 2pt minus 2pt


\title{ REsearch Log }
\author{Zijian Yao }
\date{Summer 2014}


\usepackage{fancyhdr}
\pagestyle{fancy}
\lhead{2014 Summer} %Change change change!
\rhead{Zijian Yao}
\renewcommand{\headrulewidth}{0.4pt}

\begin{document}


\section{Introduction} \indent

We are interested in the unimodality of certain statistics defined on quotients of the Boolean algebra. The first several lemmas are from Stanley's [Algebraic Combinatorics].

\lemma Let $P$ be a graded poset of rank $n$. A map $u_i: P_i \ra P_{i+1}$ is called an order matching if it is injective and satisfies  $x < \mu (x)$ for all $ x \in P_i$.  We claim that if for some $j: 0 \le j \le n$ there are order matchings: $$P_0 \ra P_1 \ra ... P_{j-1} \ra P_j \leftarrow  P_{j+1} \leftarrow ... \leftarrow P_n,$$ then $P$ is rank unimodal and Sperner. 


\lemma Suppose there is an injective linear transformation $U: \R P_i \ra \R P_{i+1} $  which is an order-raising operator, i.e., for all $x \in P_i$, $U(x)$ is a linear combination of elements $y \in P_{i+1}$ satisfying $x < y$. Then there exists an order raising-matching $\mu: P_i \ra P_{i+1}$.   

Now we apply the lemmas to Boolean algebra and its related posets. Before that, we consider the following order-raising operation (which is a natural choice): 

For each $0 \le i < n $, we define a linear transformation $$U_i: \R (B_n)_i \ra \R (B_n)_{i+1} \qquad \text{ by } U_i (x) = \sum_{y \in (B_n)_{i+1} \::\: x < y} y.$$

\lemma{\label{020204}} $U_i$ is injective for $i < n/2$ (and surjective otherwise). 

\proof See [Stanley]. \\

We slightly generalize the idea. Let $V_i^{(r)}$ be generated by basis $\{e_{(x,y)}\}$ such that $y \in P_i$ and $x \in P_{i -r}$ where $x < y$. Define $U_i ^{(r)}: V_{i}^{(r)} \ra V_{i+1}^{(r)} $ by $e_{(x, y)} = \sum_{j \notin y} e_{(x\cup \{j\}, y \cup \{j\} )}$. Then similarly to what we have above, one has that $[U_i^{(r)}]^T [U_i^{(r)}] - [U_{i-1}^{(r)}] [U_{i-1}^{(r)}]^T = (n -i ) I_i - (i-r) I_i = (n-2 i + r) I_i$. Then one can easily calculate the eigenvalues, which are again positive.  (Note that we start with $[U_{r}^{(r)}]^T [U_{r}^{(r)}]$. Now as we go up each level, we add $n-2i + r $ to each eigenvalue. )\\



Now we consider subgroup $G < S_n$, which acts on $B_n$. 

\definition A quotient poset by action $G$ on $B_n$ is $P/G : = \{G_x\}$ where elements of $P/G$ are $G$-orbits $G_x$ ordered by relation $G_x \le G_y$ iff there exists $ x' \in G_x$ and $y' \in G_y$ such that $x' \le y'$. 

Two particularly important quotients of $B_n = 2^{[n]}$ are the following: 

(1). The Necklace poset $B_n/C_n$ where $C_n$ is the cyclic group of order $n$ generated by $(1 \: 2 \: 3 \: ... \: n)$.

(2). The Gaussian poset $L(m,l)$ for some $n = m \cdot l$, which is $B_n / (S_m \wr S_l)$. The action of the wreath product does the following to an $m \times l $ filled rectangular tableau: it permutes each row freely and then swaps the rows.  

Stanley showed in 1984: 

\theorem[Stanley, 1984] For any subgroup $G \subset S_n,$ 
let $p_i = \# \{\text{G orbits Gx with} |x| = i\} = \#
\{B_n/G\},$ then the sequence $p_0,p_1,\ldots, p_n$ is both symmetric: $p_i = p_{n-i}$ and unimodal: $p_0 \le \cdots \le p_{\lfloor n/2 \lfloor} \ge \cdots \ge p_n.$

\proof The proof is to consider the $G$-invariant subspace of $V_i = V_i^{(1)}$, and realize that the map $U_i$ commutes with the action of $G$, thus can be restricted to the $G$-invariant subspace. 

\corollary[Sylvester, 1878] $L(m,l)$ defined as above has unimodal rank sequence. 


\definition  For $\lambda \in L(m,l),$ we denote the statistic $\nu(\lambda) =$ the number of distinct row sizes (or equivalently the number of corner cells).


\theorem[Pak and Panova, 2013 ] Fix $r \geq 0.$ Let sequence $p_i = \sum_{\lambda \in L(m,l),|\lambda | = i}^{}\binom {\nu(\lambda)}r.$ Then, $p_r,\ldots, p_n$ is both symmetric and unimodal, with $p_i =p_{n+r - i}.$



\newpage
\section{The wreath product} \indent 

First we introduce/recall notations. $B_n$ is the Boolean algebra on $n$ elements $\{1, ..., n\}$, $G < S_n$ is a subgroup of the symmetric group $S_n$.  Let $V_i^{(r)}$ be generated by basis elements $\{{(x,y)}\}$ such that $y \in P_i$ and $x \in P_{i -r}$ where $x \lessdot y$, for convenience we denote the set of the basis elements by $A_i$, so $$A_i = \{ (x, y) | x \in P_{i-1}, y \in P_i,  \, x \lessdot y \}, \quad \text{ and } \quad  V_i^{1} = \R A_i.$$

For each $i$ we define a relation $\sim$ on $A_i$ as follows: we say that $$(x, y) \sim (x', y') \text{ iff there exists}  \sigma, \tau \in G  \text{ such that } \sigma x = x', \tau y = y'.$$ I want to emphasize the fact that since $(x', y') \in A_i$, the relation necessarily requires $\sigma x \lessdot \tau y$, i.e., $(\tau^{-1} \sigma ) x \lessdot y$.  One can easily verify that $\sim$ is a an equivalent relation on $A_i$. Let us denote the number of such classes on $A_i$ by $p_i$, i.e., $$p_i = \# \text{ of equivalent classes formed by } \sim . $$  In the Hasse diagram of the graded poset $B_n/G$, the number $p_i$ precisely counts the number of edges doing down from rank level $i$ to rank level $(i-1)$. (And it is consistent with the notation Vic gave for the case $r = 1$).  

Now we define another equivalence relation $\equiv$ on $A_i$ by the following group action $$G \times A_i \ra A_i \quad \text{ where } \sigma (x, y) \mapsto (\sigma x, \sigma y ).$$ 
and define $q_i$ to be the number of $G$-orbits for each $i$ such that $ 1 \le i \le n$. Note that the integer $q_i$ is the dimension of $(V_i^{(1)})^G$, the invariant subspace of $V_i^{(1)}$. (This is step-by-step analogous to Stanley's proof). 

The same way that Stanley showed that $U_i$ is injective shows that the $U_i^{r}$ is injective in general. Especially we know that $U_i^{(1)}$ is injective, which implies the following lemma:

\lemma For any $G < S_n$, the statistics $q_i$ is unimodal. 

\proof See details in previous section.

\lemma For any  $G < S_n $ and all $1 \le i \le n$, we have $q_i \ge p_i$. 

\begin{proof} This is almost trivial, since each $G$-orbit lives in some $\sim$ equivalent class. 
\end{proof}

Note that we are particularly interested in the case for $G$ where $q_i = p_i$, since this would imply that the sequence $p_i$ are unimodal, as a corollary of results $4.4, 4.5$ and the exercise Vic gave, which says $U_i^{r}: V_i^{(r)} \ra V_{i+1}^{(r)}$ is injective.\\

Our first result concerns the Wreath product. 

\proposition Let $G = S_m \wr S_l$ where $n = ml$, and $p_i, q_i$ as defined above. Then $p_i = q_i$. 

\remark This gives a different proof that for group $S_m \wr S_l$, the statistics $p_i$ are unimodal, which is the case $r= 1$ in the paper by Pak and Panova. 

\begin{proof}

To prove the proposition, we need to show that each $\sim$ equivalent class is precisely a $G$-orbit (i.e., a $\equiv$ equivalent class). 

We formulate in terms of the following statement: for any pairs $(x, y), (x', y') \in A_i$ with $\sigma, \tau \in G$ satisfying $\sigma x = x', \tau y = y'$ and $\tau^{-1} \sigma x \lessdot y$, then there exists $g \in G$ such that $gx = x' $ and $gy = y'$. 

Note that it suffices to prove the statement when $\tau = 1$, i.e., if there is sigma satisfying $\sigma x \lessdot y$, then there exists $g \in G$ such that $gx = \sigma x$ and $g y = y$, since we can replace $\sigma $ and $g$ by $\tau^{-1} \sigma $ and $\tau ^{-1} g$ respectively.  We can further reduce the proof to the case where $y$ is has the shape of a young tableau for the following reason: let $\delta$ be the wreath transformation that takes $y$ to $y'$, where $y'$ is of the shape of a YT (i.e., size of the rows weakly decreases from top to bottom). Since $\sigma x \lessdot  y$, then $\delta \sigma x \lessdot \delta y = y'$. Hence, if we know that there is some $g'$ such that $g' x = \delta \sigma x $ and $g' y = y'$, then letting $g= \delta^{-1} g'$ gives us $g x = \delta^{-1} g'x = \sigma x$ and $g y = y$. 

Let us recap what we want to prove: $(x, y) \in A_i$, with $y$ in the shape of a Young tableau, and $\sigma \in G$ such that $\sigma x \lessdot y$. First consider the case when $\sigma$ only permute some rows but do not swap them. Let $x$ be formed by taking out a square $Q$ from some row $R$, note that $x \minus R = y \minus R$, so $\sigma$ fixes $y$ on all other rows except for $R$. If $\sigma y = y$, then we are done by taking $g = \sigma$, otherwise, since $\sigma x \lessdot y$, we know that $\sigma (x \cap R \minus Q) < (x \cap R) $ and $\sigma (Q) = Q'$ is a square out of $y$ in the row $R$. Let $\sigma'$ be the transformation that takes $Q'$ to $Q$ and fixes everything else, then it is clear that $\sigma' \sigma x = \sigma x$ and $\sigma' \sigma y = y$, so we found such $g = \sigma' \sigma$. Now without loss of generality we can assume that $x$ is formed by taking out a corner square $Q$ from $y$ in row $R$ and $\sigma$ swap the rows. If $\sigma$ does not affect $Q$ then again take $g = \sigma$, otherwise, since $\sigma x \lessdot y$, we know that $\sigma$ can only swap row $R$ with some other row below it, say $R'$, where $x \cap R'$ has the same size as $x \cap R$, but in this case, swapping rows do not affect the shape of $x$, so we can simply take $g = 1$ the identity. This concludes the proposition.
\end{proof}


\newpage

\section{The Necklace poset}

\subsection{Formula for $q_i$}

Now what about the Necklace poset where $G = C_n = \gr{(1 \:2\: ... \: n)}$? 

For this situation we claim the following: 

\proposition{} Let $G = C_n$, then for $i$ such that $2 \le i \le n - 1$, we have that $q_i > p_i$. In addition, we can calculate $q_i$ fairly easily in this particular case: $$q_i = {n-1 \choose i -1}.$$

Note that this gives an upper bound for $p_i$. 

\begin{proof}

To show that $q_i > p_i$ when $2 \le i \le n-1$, we consider a pair of element $(x_0, y_0)$ where $x_0 = \{1, 2, ..., i-1\}$ and $y_0 = \{1, 2, ..., i\}$. Take $\sigma $ being the generating permutation $(1 \:2\: ... \: n)$ and $\tau$ the identity. Then clearly $\sigma x = \{2, ..., i \} < y$. Now consider $g = \sigma ^d$ for some integer power $d$, if $g x = \sigma x$, then necessarily $g  = \sigma$. Since $i \le n -1$, $\sigma y = \{2, ..., i+1 \} \ne y$. This shows that the $\equiv$ equivalent class of $(x, y)$ contains more than $1$ $G$-orbit, therefore $q_i > p_i$.

Now consider the second part of the statement, first we show that the action $G \times A_i \ra A_i$ is faithful, namely for any $(x, y) \in A_i$,  $\tau (x, y) = (x, y) $ if and only if $\tau = 1$, the identity element of $G$. Note that, if $\tau x = x$ and $\tau y = y$, then $\tau (y \minus x) = y \minus x$, where $y \minus x \in P_1$ is a one element set, since $\tau = (1 \: 2 \: ... \: n)^d$  for some power $d$ (which is a rotation), $\tau$ fixes the one element set iff $\tau = 1$. Now for any $(x, y) \in A_i$, the stabilizer $\text{Stab}_{(x, y)}$ is trivial. By the orbit-stabilizer lemma, we know the orbit $G_{(x,y)}$ of $(x, y)$ always contains $|G| = n$ elements.  It is easy to show that $|A_i| = {n \choose i } \times i$, so $q_i = |A_i|/|G_{(x, y)}| = {n \choose i } \times  \frac{i}{n} = {n -1 \choose i -1}$. 

\end{proof}



\subsection{$p_i$ is unimodal when $n$ is prime in $B_n$}

Now we have (at least) two jobs to do, first to determine for which groups $G$ one does have $p_i = q_i$, secondly, to show that $p_i$ is unimodal for $G = C_n$ and other groups. 

For the second part with $G = C_n$, I compared the list of $(p_1, ..., p_n)$ with $(q_1, ..., q_n)$ for a few cases (this could lead to a proof by realizing how $q_i$ differs from $p_i$, i.e., how $G$-orbits group to form the $\equiv$ classes). 


Here are the experimental data: $i$ starting from $1$ to $n$. 

* n =3, $\{ p_i \} = \{1 , 1, 1\}; \{q_i \} = \{1, 2, 1\}$ 

* n =4, $\{ p_i \} = \{1 , 2, 2, 1\}; \{q_i \} = \{1, 3, 3, 1\}$ 

* n =5, $\{ p_i \} = \{1 , 2, 4, 2, 1\}; \{q_i \} = \{1, 4, 6, 4, 1\}$ 

* n =6, $\{ p_i \} = \{1 , 3, 9, 9, 3, 1\}; \{q_i \} = \{1, 5, 10, 10, 5, 1\}$ 

* n =7, $\{ p_i \} = \{1 , 3, 12, 17, 12, 3, 1\}; \{q_i \} = \{1, 6, 15, 20, 15,6, 1\}$ 
$$...... $$

I was hoping that this comparison would point to some directions, and it does, at least to some extend. Notice that for $n= 3, 5, 7$, we have that $q_i - p_i = (n-1)/2$ for $ 2 \le i \le n-1$. So I  think that this is true for all $n$ = primes. 

\proposition Let $G = C_n$ where $n$ is a prime, then $q_i - p_i = (n-1)/2$ for $ 2 \le i \le n-1$. 

\begin{proof} 

The idea is to group the $G$-orbits to form $\sim$ classes, in fact, we count the number of $G$-orbits  in each $\sim$ classes. Note that $n$ being a prime guarantees that the action of any nontrivial $\sigma \in C_n$ has no fixed points, since $\sigma$ is always an $n$-cycle in its cycle decomposition. Now suppose $(x, y) \in A_i$  is a pair such that $\sigma x \in y$ for some no trivial $\sigma$, then there is no $g \in C_n$ such that $gx = \sigma x$ and $g y = y$, since $gx = \sigma x \so g = \sigma $ and $g y = y \so g = 1$. Let $\mathcal O$ be a $\sim$ class with representative $(x, y)$, then by the argument above, the number of distinct $\sigma \in G$ such that $\sigma x < y$ is precisely the number of $G$-orbits in $\mathcal O$, in particular, if the only $\sigma $ such that $\sigma x < y$ is identity, then $\mathcal O$ is a $G$-orbit. 

Therefore, the problem of counting $q_i - p_i$ is to count the number of distinct $G$-orbits $G_{(x, y)}$ with distinct $\sigma \in G$ such that $\sigma x < y$, and we claim that regardless of $i$, this number is $(n-1)/2$.  

We prove the claim using a combinatorial method, we consider rotations of the necklace pattern of the $G$-orbits of $y$, where $y \in P_i$. We label the ``pearls'' - empty cells - of the necklace by $1, 2, ..., n$. For example, to represent $\{1, 2, 3\} \equiv \{2, 3, 4 \} ...$, we simply fill $3$ consecutive pearls to be solid, others would empty. Let $\sigma_0 = (1 \: 2 \: ... \: n)$ be the generator of $C_n$, which is a $1$-click rotation. 

First consider $\sigma = \sigma_0$ and $\sigma x < y$, i.e., we take out a filled pearl from $y$, and rotate $y$ by a $1$-click rotation, the remaining filled pearls fits in $y$. It is clear that the only possibility for this to occur is we have $n$ consecutive filled cells, i.e., $y = \{1, 2, ..., i\}$ and $x = \{1, 2, ..., i-1 \}$ (up to actions by $G$). 

Now consider $\sigma = \sigma_0^2$ and $\sigma x < y$,  i.e., we take out a filled pearl from $y$, and rotate $y$ by a $2$-click rotation, the remaining filled pearls fits in $y$. Now we again trace back from the ``last'' filled pearl in $y$ (any filled pearl could be the last one since we are in a circle), say at position $P$, then at position $P - 2, P -4, ... $ (reduce mod $n$ whenever necessary) there need to be a filled pearl. In this case, there is again only one possibility, namely $G$-orbits of $\{1, 3, ... 2i -1\}$  and $x = \{1, 3, ..., 2i -3\}$. 
  
Similarly, for any $\sigma = \sigma_0^{j}$, $j < n/2$, there is a precisely one possibility for $\sigma x < y$.  Notice that for $j > n/2$, the necklace $\{1, j+1,2j+1, ...  \}$ and $\{1, (n-j)+1, 2(n-j)+1, ... \}$ are the same necklaces, so we do not count them again. It is also easy to show that, for all distinct $j < n/2$, the necklaces $\{1, j+1, 2j+1, ..., (i-1)j+1\}$ are all distinct, since $n$ is a prime number. 

This proves the claim.
\end{proof}


As a corollary we have the following theorem: 

\theorem Let $G = C_n$ where $n$ is a prime, and $p_i$ as defined above. Then the sequence $p_i$ is unimodal. 

\begin{proof}

By the proposition and the symmetry of $p_i$, we only need to show $p_1 \le p_2$, but $p_1 = 1$, so we are done.
\end{proof}


\remark Note that the proof of the proposition above shows that, if the group $G$ contains $C_n$ as a subgroup where $n$ is prime, then $p_i = q_i$ if $G$ is generated by $C_n$ with arbitrary reflections. Namely, if $G = D_n$, the Dihedral group on of order $2n$, then $p_i = q_i$.    

We include this result as a proposition (along with \ref{020208})

\proposition{\label{020214}}  Let $G = D_{n}$ where $n$ is prime, and $p_i, q_i$ as defined above. Then $p_i = q_i$. 



\subsection{$p_i$ is unimodal in general}


The same method can be used to bound $q_i - p_i$ for $n$ not necessarily prime.  We care about the upper bound of the difference, since we have a precise formula for the difference of $q_{i} - q_{i-1}$, so by bounding the difference of $q_i - p_i$, one hopes to show the unimodality of $p_i$ indirectly.


\lemma{\label{020602}} Let $G = C_n$, and $p_i, q_i$ be defined as in Day 2. Let $\lambda_i = q_i - p_i$, then for sufficiently large $n$ and $i$, the difference $\lambda_{i+1} - \lambda_{i}$ is bounded above by some (will determine) polynomials of $n$ to some degree not too large. \\

\begin{comment}
First let us consider an ideal case, where we can bound $\lambda_i = q_i - p_i$.

\lemma Let $G$ be a subgroup of $S_n$, and let $p_i, q_i, \lambda_i$ be defined as above. Suppose that $\lambda_i = q_i - p_i \le n/2$, then $p_i$ is unimodal. 

\begin{proof} We prove the lemma for sufficiently large $n$ and $i$, we require $n\ge 7$ and $i \ge 3$.  We want to show that $p_i \ge p_{i-1}$ for $2 \le i \le (n+1)/2$. Note that $p_1 = 1$, so only need to show the claim for $3 \le i$. 

First consider when $3 \le i \le  n/2$.  From lemma \ref{020602}, we only need to prove that $q_i - q_{i-1} \ge n/2$, which would imply that $p_i \ge q_i - n/2 \ge q_{i-1}  \ge p_{i-1}$. 

Recall that $q_i =\dstyle {n-1 \choose i - 1}$, so $$q_i - q_{i-1} = \frac{(n-1)!}{(i-1)! (n-i)!} - \frac{(n-1)!}{(i-1)!(n-i+1)!}= {n \choose i -1} \frac{n-2i +2}{n} $$

Note that $n - 2i \ge 0$ and $i \ge 3$ by assumption, so $$q_i - q_{i-1} \ge {n \choose i -1} \cdot \frac{2}{n} \ge {n \choose 2} \cdot \frac{2}{n} = n-1 \ge \frac{n}{2}.$$

The remaining case is when $n$ is odd and $k = (n+1)/2$. By assumption we know that $k \ge 4$. This time we have   
$$q_k - q_{k-1} \ge {n \choose k -1} \cdot \frac{2}{n} \ge {n \choose 3} \cdot \frac{2}{n} = \frac{(n-1)(n-2)}{6} \ge \frac{n}{2}.$$
This concludes the unimodality of $p_i$ for this specific case.

\end{proof}

\end{comment}


Remark: In the beginning I tried to bound the difference $\lambda_i = q_i - p_i$ directly, due to a calculation error, that does not seem to work out as nicely. Aaron used the same method and the bound actually works. So we do not need to bound $\lambda_{i+1} - \lambda_i$. But it proves the theorem as well. \\ 

Nevertheless I shall write up my solution: \\

The idea is to bound $\lambda_{i+1} - \lambda_{i}$ by some polynomial $F(n)$ of some degree $d$, which is relatively small compared to $n$.  Since
$p_{i+1} - p_{i} = (q_{i+1} - \lambda_{i+1}) - (q_i - \lambda_i) = (q_{i+1} - q_i) - (\lambda_{i+1}- \lambda_i) \ge ( q_{i+1} - q_{i} ) - F(n)$, as long as $F(n)$ can be bounded by the difference of $q_{i+1} - q_{i}$ [in a similar fashion as shown above] then we can still say something about the unimodality of $p_i$, at least for specified $i$ and $n$. 

We introduce an extra piece of notation, let $B_m/C_m$ be the necklace quotient poset, and let $q (i, m)$  be the size of $C_m$-orbits of pairs $x \lessdot y$ in $B_m \times B_m$ such that $|y|=i$, note that this is the $q_i$-statistics for $B_m$, the Boolean algebra on $m$ elements.  Similarly we define $p (i, m)$.  Later we shall generalize results to $p(i, m ,r)$ for $r \ge 2$, analogous to what Pak and Panova did. 


We need more notations for the case where $G = C_n$. Consider an element $\sigma \in G$ acting on $x \lessdot y$. We regard $x \lessdot y$ as a way to colour the necklace with $n$-positions numbered cyclically (say all elements in $x$ are coloured black, and the extra element in $y$ is coloured white). The effect of the action by $\sigma$ is to rotate the necklace. suppose that $\sigma x \lessdot y$, then we call each subset of $x$ that is fixed by $\sigma$ a \textbf{full cycle}, and $x \minus \text{ all full cycles }$ is called a \textbf{tail cycle}.  The language makes intuitive sense regarding the picture of the necklace that we have in mind. 

\lemma Let $G= C_n$, $q_i$ and $p_i$ be defined as usual,  then $$q_i - p_i = \#\{G_{(x,y)} : \exists \: \, \sigma_0^r x \lessdot y \text{ where } r < n/2, \text{ but } \nexists \:  g \text{ s.t. } g x = \sigma_0 ^r x, g y = y \}$$ where $\sigma_0 = (1 \: 2 \: ... \: n)$ is the generating $1$-click permutation.

\begin{proof}

First we prove a sublemma.  \\

\textbf{sublemma:} Let $G = C_n$, let $\sigma_0$ be as in the lemma, assume that $\sigma_0^r x \lessdot y$ for some pair $x \lessdot y$ and $r < n/2$, and there is no $g$ such that $g x = \sigma_0^r (x)$ and $gy = y$, then $r$ is the only integer $1 \le r < n/2$ such that $\sigma_0^r (x) \lessdot y$.  \\

By the assumption we know that $x$ has a tail cycle under the action of $\sigma_0^{r}$, for otherwise $x$ is fixed and we can simply take $g = id$. We also know that the tail cycle is not completed to be a full cycle when we add the extra element $y \minus x$, for otherwise $y$ is fixed under $\sigma_0^r$  and we can take $g = \sigma_0^r$.  Call this tail cycle $\gamma$, so $\gamma$ has some black spots and a white spot, which is $z = y \minus x$. Let $\sigma$ be any rotation such that $\sigma x \lessdot$, then $\sigma$ has to take a full cycle of $x$ (under $\sigma_0^r$) to another full cycle (under $\sigma_0^r$), so $\sigma (\gamma \minus z) \lessdot \gamma$. Recall that $\sigma_0^r (\gamma \minus z) \lessdot \gamma$ by assumption, so the elements in $\gamma$ are $r$-positions away from each other, namely $z, z-r, z-2r, ... z -l r$, but $z+r$ is empty since the cycle is assumed to be a tail cycle. Since $\sigma (\gamma \minus z) \lessdot \gamma$, $\sigma$ has to send something to $z$, say $z - m r$ is sent to $z$, then $z - {m-1}r$ is sent to $z+r$, which is a contradiction. This proves the sublemma. \\

Now we return to the lemma. The difference of $q_i - p_i$ is counting the difference of the number of $G$-orbits and the number of edge classes, i.e.,  the $\sim$ classes. It is clear that the $G$-orbits correspond bijectively to the pattern of pairs of $(x \lessdot y)$ in the necklace picture, i.e., the pattern where we colour $i$ elements in the  $n$-element cycle, where $1$ of them is coloured white and the rest $i-1$ black (the black ones correspond to $x$), note that the pattern is uniquely determined by each $G$-orbit up to rotational equivalence. The sublemma tells us that each edge equivalence class contains at most two $G$-orbits, and the number is precisely the number of a pair of $G$-orbits $G_{(x \lessdot y)}$ and $G_{(\sigma x \lessdot y )}$ such that they are distinct $G$-orbits. We restrict $r < n/2$  to count the number of such pairs, since $(n -r)$-click rotation will simply send $(\sigma_0^r \, x \lessdot y)$ to $(x \lessdot y)$. This proves the lemma, and provides a way to count $\lambda_i = q_i - p_i$, though practically an upper bound suffices to show unimodality. 
\end{proof}



Next we give a formula to compute $\lambda_{i+1} - \lambda_{i}$ for the group $G = C_n$, where $n$ is arbitrary. 

\proposition Let $B_n$ be the Boolean algebra, $C_n$ our usual cyclic group, then $$\lambda_{i+1} - \lambda_i \le \sum_{k | (n , i);  3 \le k } q (\frac{n}{k}, \frac{i}{k}) $$

\begin{proof}

First, we note that, if $\sigma_0 = (1 \: 2 \: ... \: n)$ is the generator of $C_n$, then
 $$q_i - p_i = \#\{G_{(x,y)} : \exists \: \, \sigma_0^r x \lessdot y \text{ where } r < n/2, \text{ but } \nexists \:  g \text{ s.t. } g x = \sigma_0 ^r x, g y = y \}$$ 

Therefore, to count $q_i - p_i = \lambda_i $, we only need to count, for each $r$-click rotations where $r \le n/2$, the number of distinct $G$-orbits $G_{x, y}$ such that $ \sigma_0^r x \lessdot y, \, \, \text{ but  there does not exists } g \in C_n \text{ s.t. } g x = \sigma_0 ^r x, g y = y$. For convenience, call such an $G$-orbit a \textbf{special} orbit.  

Note that, for each $r \le n/2$ such that $(r, n)$, there is precisely one such $G$-orbit, since it this case $\sigma_0^r$ is a full $n$-cycle [See proof of proposition \ref{020220}]. This holds regardless of $i$. 

Now consider $r \le n/2$ such that $(r, n) > 1$. In this case, $\sigma_0^r$ fixes elements of the form $\{s, s+r, s+2r, ..., \}$ for any starting point $s$, which are proper subsets of $[n]$. In this case, $\sigma_0^r$ partitions G-orbits $G_y$ into many sub-orbits, with all but at most $1$ of the orbits being one of those fixed elements of $\sigma_0^r$, i.e., a smaller cycle $s \ra (s+r) \ra (s+2r) \ra ... \ra s$, since $\sigma_0^r x \lessdot y$ [So one cannot have more than $1$ of the sub-orbits being a non-fixed element]. Now, the $y$ which are partitioned into all fixed elements (smaller cycles) do not contribute in counting $\lambda_i$, since $\sigma_0^r$ fix such $y$. Next consider the case where one of the sub-orbits of $y$ partitioned by $\sigma_0^r$ is not fixed by the action, then this would contribute to add $1$ when we count $\lambda_i$. 

Now let us focus on the difference of $\lambda_{i+1} - \lambda_{i}$, note that the difference only occurs when we count $r \le n/2$ and $(r, n) > 1$. Let $G_{(x, y)}$ be a $G$-orbit where $x \lessdot \, y$ such that $|y| = i$, and let $\sigma_0^{r} (x) \lessdot  y$ and suppose that there is no $g$ such that $g x = \sigma_0 ^r (x)$ and $gy = y$. From the discussion above, we know that there is a sub-orbit of $y$ which is a noncyclic sequence $z := s \ra (s+r) \ra (s+2r) \ra ... \ra (s+ lr)$  for some $l $, and $x = y \minus \{l\} $. Suppose that adding $s-r $ (mod $n$ whenever necessary throughout the proof) does not complete the cycle for  $z \cup \{(s - r)\}$, then if we let  $y' = y \cup \{(s - r) \} $ and $x' = x \cup \{(s -r)\}$, then the orbit $G_{(x', y')}$ where $|y'| = i+1$ is also a special orbit, thus does contribute $1$ when we count $\lambda_{i+1}$.  In this case, our counting process for $\lambda_{i+1}$ and $\lambda_{i}$ both increase by $1$, so such special orbit does not contribute to $\lambda_{i+1} - \lambda_{i}$. Now suppose that adding $s -r$ completes the sequence $z$ into a small cycle, namely $s \ra (s+r) \ra ... \ra (s-r) \ra s$. Now $y'$ will not be a special orbit anymore, since $\sigma_0^r $ fixes $y'$. In this situation,  we increase  $1$ for $\lambda_i$ only. If we choose to ignore this scenario, we will get an upper bound for $\lambda_{i+1} - \lambda_i$. 

Now return to the situation where $y$ is partitioned in disjoint cycles, so $G_{(x,y)}$ is not special, but adding $1$ arbitrary point to enlarge $y$ into $y'$ will yield a special orbit $G_{(x', y')}$ where $x' = y$ and $y' = y \cup \{s'\}$ for any $s' \notin y$. We need to count how many times this happens.  First we note that this happens when $d | n$, and $(n/d) | i$ where $d = (r, n)$. Note that, for different $r$ such that $(r, n)$ remains the same, we only result in $1$ family of such special orbits, since the $x$ part will be $d$-click rotationally symmetric. We claim that , this is at most counting,  over all $d | n$, where $1 < d < n$ and $(n/d) | i $, the sum $\dstyle \sum q (\frac{id}{n}, d)$. This is because at most we have $q(\frac{id}{n}, d)$ number of ways to insert the tail cycle (with precisely $1$ coloured white) in the block of $d$ elements. 

We can introduce $k = n/d$  and rewrite the sum as 
$$\sum_{k | (i, n), \, 1 < k < n} q(\frac{i}{k}, \frac{n}{k}).$$ Note that in our sum, $k$ can never be $2$, since other wise the tail cycle of $x$ has precisely $1$ element, and then $y$ is fixed (consisting of full cycles). So this proves the assertion that $$\lambda_{i+1} - \lambda_i \le \sum_{k | (n , i);  3 \le k } q (\frac{i}{k}, \frac{n}{k}). $$ 
 

Note that, from the process of calculating $\lambda_{i+1} - \lambda_{i}$, we can potentially obtain a better bound if we need to. 
\end{proof}

\corollary For $G= C_n$, the $p_i$ are unimodal. 

\begin{proof}

To prove this, we only need to bound $ \dstyle \sum_{k | (n , i);  3 \le k < n/2 } q (\frac{i}{k}, \frac{n}{k}) $ as above. It is clear that for sufficiently large $n$, say for $n \ge 9$, $$q(i/k, n/k) = {n/k - 1 \choose i/k -1} \le {\lceil n/3-1 \rceil \choose \lceil i/3 -1 \rceil },$$ where at most there are $i$ of those terms, and $ i \ge n/2$, so we coarsely bound the sum by $$\frac{n}{2} \cdot {\lceil n/3-1 \rceil \choose \lceil i/3 -1 \rceil }. $$ We want to show that this is smaller than the difference of $$q_i - q_{i-1} = \frac{(n-1)!}{(i-1)! (n-i)!} - \frac{(n-1)!}{(i-1)!(n-i+1)!}= {n \choose i -1} \frac{n-2i +2}{n} \ge  {n \choose i -1} \frac{2}{n} .$$

Namely, we want to show that, for sufficiently large $n$, 

$$ (\frac{n}{2})^2 \cdot {\lceil n/3-1 \rceil \choose \lceil i/3 -1 \rceil }  \le   {n \choose i -1}.$$

This is an easy bound. We check by hand (computer) that the statement is true for $n \le 12$, and for $n > 13$, the bound already works.
\end{proof}


\newpage
\section{The Dihedral Group} \indent


A similar method also works for $G = D_{2n}$ the dihedral groups, which acts naturally on $B_n$. 

We define $q_i = q (i, n)$  and $p_i = p(i, n)$ similarly. 

\proposition We have an explicit formula for $q_i$: 
$$q_i = \frac{1}{2} \Big( {n-1 \choose i -1} + \frac{1}{2} [(-1)^{n(i+1)}+1] \cdot { \lceil n/2\rceil -1  \choose \lceil i/2 \rceil - 1}    \Big)$$


\begin{proof}

Consider an element $x \lessdot y$ and its stabilizer $\text{Stab}{(x \lessdot y)}$, note that any element $\tau $ that fixes both $x$ and $y$ has to fix the difference $y\minus x$, which is a one-element set. Since $G = D_n$, the only elements that fixes $y \minus $ are the reflection about the symmetry line through the element $y\minus x$ and the identity. So $|\text{Stab} (x \lessdot y)| = 1 $ or $2$, depending on what $x$ and $y$ are. From the stabilizer-orbit theorem, we know that each stabilizer corresponds to an orbit under the $G$-action. Let $\mu_1$ be the number of $G$-orbits with the trivial stabilizer and $\mu_2$ be the number of $G$-orbits with the stabilizer of size $2$, which contains the identity and a reflection. By the stabilizer-orbit theorem, each orbit with the trivial stabilizer is of size $|D_{2n}| /1 = 2n $, while all other orbit is of size $|D_{2n}|/2 = n$.  This tells us, among other things, that $$\mu_1 \cdot 2n + \mu_2 \cdot n = |\{(x \lessdot y)\}| = {n \choose i}  \cdot {i \choose 1} = n {n-1 \choose i -1}.$$

Our goal is to calculate $q_i = \mu_1 + \mu_2$. Our previous equation implies: $$ q_i = \mu_1 + \mu_2= \frac{1}{2} \Big[ (2 \mu_1 + \mu_2) + \mu_2   \Big] = \frac{1}{2} \Big[ {n -1 \choose i -1} + \mu_2   \Big].$$

It remains to calculate $\mu_2$, which amounts to count the number of $G$-orbits such that the reflection also fixes $x$. Label all the elements of $B_n$ in a cyclic order along a circle (with the ``necklace'' picture in mind). Since we are counting the number of $G$-orbits, where all rotations are allowed, we can without loss of generality assume that $y\minus x = \{1\}$, $\mu_2$ is precisely the number of ways to insert the rest of $i-1$ elements into the cycle such that they are symmetric about the line through $1$ and the position $(n+1)/2$.  We break into cases:

* If $n$ is odd and $i$ is even, then there is no such a way to put the $i-1$ elements symmetrically (in the way we want) among the $n-1$ spots. In this case $\mu_2=0$, so we add a factor of $\frac{1}{2} ((-1)^{n(i+1)} + 1)$ to cover this case. 

* For all other cases, there are precisely $\lceil n/2 -1\rceil$ spots to insert $\lceil i/2 - 1 \rceil$ elements. Note that $1$ of the elements might have to go to the position $(n+1)/2$ when $n$ is even, the ways to insert $x$ is determined by the positions of the rest $\lceil i/2 - 1 \rceil$ elements (because of the required symmetry). 

This gives the desired formula $\dstyle \mu_2 =  \frac{1}{2} [(-1)^{n(i+1)}+1] \cdot { \lceil n/2\rceil -1  \choose \lceil i/2 \rceil - 1}   $. Therefore, 
$$q_i = \frac{1}{2} \Big( {n-1 \choose i -1} + \frac{1}{2} [(-1)^{n(i+1)}+1] \cdot { \lceil n/2\rceil -1  \choose \lceil i/2 \rceil - 1}    \Big)$$
\end{proof}

\corollary For $G = D_{2n}$, $p_i$ are unimodal. 

\proof The proof is line by line analogous to the case where $G = C_n$, the previous bound is a bigger (worse) bound for this case, but it works already.
























































































\end{document}
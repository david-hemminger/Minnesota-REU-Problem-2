%%%%%%%%%%%%%%%%%%%%%
%   AMS packages    %
%%%%%%%%%%%%%%%%%%%%%
\documentclass[pdf]{beamer}
\usepackage{amsmath}
\usepackage{amsxtra}
\usepackage{amscd}
\usepackage{amsthm}
\usepackage{amsfonts}
\usepackage{amssymb}
\usepackage{eucal}
\usepackage[all]{xy}
\usepackage{graphicx}
\usepackage{comment}
\usepackage{amssymb}
%\usetikzlibrary{matrix,arrows,decorations.pathmorphing}

\newtheorem{thm}{Theorem}
\newtheorem{cor}[thm]{Corollary}
\newtheorem{lem}[thm]{Lemma}
\newtheorem{prop}[thm]{Proposition}
\newtheorem{ex}[thm]{Exercise}
\newtheorem{conjecture}{Conjecture}
%\newtheorem*{conjecture*}{Conjecture}

\theoremstyle{remark}
\newtheorem{rem}[thm]{Remark}
\newtheorem{eg}[thm]{Example}

%\newtheorem{counterexample}[thm]{Counterexample}
\newtheorem{defn}[thm]{Definition}
%\newtheorem{claim}[thm]{Claim}
%\newtheorem{note}[thm]{Notation}
%\newtheorem{warning}[thm]{Warning}
%\newtheorem{variant}[thm]{Variant}
%\newtheorem{question}[thm]{Question}
%\newtheorem{construction}[thm]{Construction}
%\newtheorem{terminology}[thm]{Terminology}
%\newtheorem{convention}[thm]{Convention}

\newcommand\nc{\newcommand}
\nc\on{\operatorname}
\nc\renc{\renewcommand}
\newcommand\ssec{\subsection}
\newcommand\sssec{\subsubsection}
\newcommand\bO{{\mathbf O}}
\newcommand\CC{{\mathcal C}}
\newcommand\BN{{\mathbb N}}
\newcommand\BC{{\mathbb C}}
\newcommand\BF{{\mathbb F}}
\newcommand\BR{{\mathbb R}}
\newcommand\BQ{{\mathbb Q}}
\newcommand\BBZ{{\mathbb Z}}
\newcommand\uR{\underline{R}}
\newcommand\uZ{\underline{\BBZ}}
\newcommand\CF{{\mathcal F}}
\newcommand\uCF{\underline{{\mathcal F}}}
\newcommand\BZ{{\mathbb Z}}
\newcommand\BA{{\mathbb A}}
\newcommand\BP{{\mathbb P}}
\newcommand\fa{{\mathfrak a}}
\newcommand\fp{{\mathfrak p}}
\newcommand\fq{{\mathfrak q}}
\newcommand\fm{{\mathfrak m}}
\newcommand\pt{\mathrm{pt}}
\newcommand\rk{\operatorname{rk}}
\newcommand{\so}{\Rightarrow}
\newcommand{\upo}{\mathring}
\newcommand{\ra}{\rightarrow}
\renewcommand{\iff}{\Leftrightarrow}
\newcommand{\minus}{\backslash}
\renewcommand{\vec}[1]{\overrightarrow{#1}}
\renewcommand{\v}{\mathbf}
\newcommand{\gr}[1]{\langle {#1} \rangle}
\newcommand{\dstyle}{\displaystyle}
\newcommand\Aut{\operatorname{Aut}}
\nc{\bd}{\mathbf{d}}
\nc{\Hom}{\on{Hom}}
\nc{\End}{\on{End}}
\nc{\Spec}{\on{Spec}}
\nc{\Reg}{\on{Reg}}
\nc{\Specm}{\on{Specm}}
\nc\ol{\overline}
\nc\wt{\widetilde}
\nc{\one}{{\mathbf{1}}}
\renc{\mod}{\on{-mod}}
\newcommand{\id}{\mathrm{id}}
\nc{\ul}{\underline}
\nc{\uHom}{\ul\Hom}
\nc{\tHom}{\ul\uHom}
\nc{\wh}{\widehat}
\nc{\Vect}{\on{Vect}}
\nc{\Res}{\on{Res}}
\nc{\Ind}{\on{Ind}}



\newcommand\fbn{\mathcal H}
\newcommand \edgequot{\text{edge-quotient bijective }}

\def\Stab{\operatorname{Stab}}
\def\Fix{\operatorname{Fix}}
\newcommand\im{\text{Im}}


\AtBeginSection[]{
	\begin{frame}{Table of Contents}
		\tableofcontents[currentsection]
	\end{frame}
}

\mode<presentation>{}
\usetheme{Copenhagen}
\usecolortheme{seahorse}

%Title slide info goes here
\title{Peckness of Edge Posets}
\author{David Hemminger, Aaron Landesman, Zijian Yao}

\begin{document}

%Title Frame
\begin{frame}
	\titlepage
\end{frame}


\section{Background}

\begin{frame}{Basic Definitions}
\begin{defn}
Let $P$ be a finite graded poset of rank $n$, that is:
\begin{itemize}
\item Elements of $P$ are a disjoint union of $P_0,P_1,\ldots,P_n$, called the \textit{ranks}

\item If $x\in P_i$ and $x\lessdot y$, then $y\in P_{i+1}$

\item Define $\rk(x) = k$, where $x\in P_k$.
\end{itemize}
\end{defn}

\begin{defn}
A map $f\colon P\rightarrow Q$ is a \textit{morphism} from $P$ to $Q$ if $x\le_P y \implies f(x)\le_Q f(y)$ and $\rk(x) = \rk(f(x))$.  We say that $f$ is \textit{injective/surjective/bijective} if it is an injection/surjection/bijection from $P$ to $Q$ as sets.
\end{defn}
\end{frame}

\begin{frame}{Peck Posets}

\begin{defn}
Write $p_i = |P_i|$.  P is

\begin{itemize}

\item \textit{Rank-symmetric} if $p_i = p_{n-i}$ for all $1\le i\le n$

\item \textit{Rank-unimodal} if for some $0\le k\le n$ we have
$$p_0\le p_1\le \ldots \le p_k \ge p_{k+1} \ge\ldots \ge p_n$$

\item \textit{$k$-Sperner} if no disjoint union of $k$ antichains (sets of pairwise incomparable elements) in $P$ is larger than the disjoint union of the largest $k$ ranks of $P$

\item \textit{Strongly Sperner} if it is $k$-Sperner for all $1\le k\le n$.

\item \textit{Peck} if $P$ is rank-symmetric, rank-unimodal, and strongly Sperner.
\end{itemize}
\end{defn}
\end{frame}



\begin{frame}

\begin{defn}
Let $V(P)$ and $V(P_i)$ be the complex vector spaces with bases $\{x |x\in P\}$ and $\{x |x\in P_i\}$
\end{defn}

\begin{lem}[Stanley, 1980]
$P$ is Peck if and only if there exists an linear transformation $U\colon V(P)\rightarrow V(P)$ such that
\begin{itemize}
\item For every basis element $x\in P$, 
$$U(x) = \sum_{y\gtrdot x} c_{x,y}y$$

\item  For all $0\le i < \frac{n}{2}$, the map $U^{n-2i}\colon V(P_i)\rightarrow V(P_{n-i})$ is an isomorphism.
\end{itemize}
\end{lem}
\end{frame}

\begin{frame}
\begin{defn}
If the Lefschetz map defined by

$$L(x) = \sum_{y\gtrdot x} y$$

satisfies the second condition in the previous lemma, then $P$ is \textit{unitary Peck}.
\end{defn}
\end{frame}

\section{The Edge Poset and a Conjecture}

\begin{frame}{Definition of the Edge Poset}
\begin{defn}
\label{defn:functor_of_edges}
For $P$ a finite graded poset, it's \textit{edge poset} $\mathcal{E}(P)$ is the finite graded poset defined as follows. 
\begin{itemize}

\item Elements of $\mathcal{E}(P)$ are ordered pairs $(x,y)\in P\times P$ where $x\lessdot y$

\item Define $(x,y) \lessdot_{\mathcal{E}} (x^\prime,y^\prime)$ if $x\lessdot_P x^\prime$ and $y\lessdot_P y^\prime$

\item Define $\le_{\mathcal{E}}$ to be the transitive closure of $\lessdot_{\mathcal{E}}$

\item Define $\rk_{\mathcal{E}}(x,y) = \rk_P(x)$.
\end{itemize}
\end{defn}
\end{frame}

\begin{frame}{A Conjecture on the Peckness of Edge Posets}
\begin{defn}
The \textit{boolean algebra of rank $n$} is the poset whose elements are subsets of $[n]$ with order given by containment, i.e. for $A,B\in B_n$, $A\le B$ if $A\subseteq B$.
\end{defn}


\begin{conjecture}[Hemminger, Landesman, and Yao 2014]
Let $G\subseteq \Aut(B_n)$.  Then $\mathcal{E}(B_n/G)$ is Peck.
\end{conjecture}
\end{frame}

\section{Main Result}
\subsection{Statement of Theorem}
\begin{frame}{Main Result}
\begin{defn}
A group action of $G$ on $P$ is \textit{cover transitive} if whenever $x,y,z\in P$ such that $x\lessdot z$, $y\lessdot z$, and $y\in Gx$, there exists some $g\in \Stab_G(z)$ such that $g\cdot x = y$.
\end{defn}

\begin{thm}[Hemminger, Landesman, and Yao 2014]
If a group action of $G$ on $B_n$ is cover transitive, then $\mathcal{E}(B_n/G)$ is Peck.
\end{thm}
\end{frame}

\subsection{Outline of Proof}

\begin{frame}
\begin{defn}
Given a group action of $G$ on $P$, we define a group action of $G$ on $\mathcal{E}(P)$ by letting $g\cdot (x,y) = (g\cdot x,g\cdot y)$ for all $g\in G$.
\end{defn}
\pause
\begin{prop}
The map $q\colon \mathcal{E}(P)/G\rightarrow \mathcal{E}(P/G)$ defined by $q(G(x,y)) = (Gx,Gy)$ is a surjective morphism.  Furthermore, $q$ is also injective if and only if the action of $G$ on $P$ is cover transitive.
\end{prop}

\begin{lem}
If $f:P\rightarrow Q$ is a bijective morphism and $P$ is Peck then $Q$ is Peck.
\end{lem}
\end{frame}

\begin{frame}
\begin{thm}[Stanley, 1984]
If $P$ is unitary Peck and $G\subseteq\operatorname{Aut}(P)$, then $P/G$ is Peck.
\end{thm}

It then suffices to show that $\mathcal{E}(B_n)$ is unitary Peck.  Unfortunately while this is true, the only proof that we know is long and computational.  Instead we outline a nicer --albeit less direct-- proof.

\end{frame}


\begin{frame}{Definition of $\mathcal{H}(P)$}
\begin{defn}
For $P$ a finite graded poset, define the graded poset $\mathcal H(P)$ as follows.
\begin{itemize}
\item Elements are pairs $(x,y)\in P\times P$ such that $x\lessdot y$

\item Define $(x,y) \lessdot_{\mathcal H} (x^\prime,y^\prime)$ if $x \lessdot_P x^\prime,y\lessdot_P y^\prime$ \textbf{and} $\mathbf{y \ne x^\prime}$

\item Define $\leq_{\mathcal H}$ to be the transitive closure of $\lessdot_{\mathcal H}$

\item Define $rk_{\mathcal H}(x,y) = rk_P(x).$
\end{itemize}
\end{defn}

\end{frame}

\begin{frame}{$\mathcal{H}(B_n)$ is unitary Peck}
add figure
\end{frame}

\begin{frame}
\begin{defn}
As before, for $G$ acting on $P$, define $g\cdot (x,y) = (g\cdot x,g\cdot y)$.
\end{defn}

\begin{rem}
Since $\mathcal{E}(P)$ and $\mathcal{H}(P)$ have the same elements and $(x,y)\le_{\mathcal{H}} (x^\prime,y^\prime) \implies (x,y)\le_{\mathcal{E}} (x^\prime,y^\prime)$, there is a natural bijective morphism $\mathcal{H}(P)/G\rightarrow \mathcal{E}(P)/G$.
\end{rem}

\begin{proof}[Proof of Main Result]
$\mathcal{H}(B_n)$ unitary Peck $\implies \mathcal{H}(B_n)/G$ Peck $\implies \mathcal{E}(B_n)/G$ Peck $\implies \mathcal{E}(B_n/G)$ Peck.
\end{proof}
\end{frame}



\end{document}
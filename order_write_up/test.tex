\RequirePackage{fix-cm}
%
%\documentclass{svjour3}                     % onecolumn (standard format)
%\documentclass[smallcondensed]{svjour3}     % onecolumn (ditto)
\documentclass[smallextended]{svjour3}       % onecolumn (second format)
%\documentclass[twocolumn]{svjour3}          % twocolumn
%
\smartqed  % flush right qed marks, e.g. at end of proof
%
\usepackage{graphicx}
%
% \usepackage{mathptmx}      % use Times fonts if available on your TeX system
%
% insert here the call for the packages your document requires
%\usepackage{latexsym}
\iffalse
\usepackage{amsmath}
\usepackage{amsxtra}
\usepackage{amscd}
\usepackage{amsthm}
\usepackage{amsfonts}
\usepackage{amssymb}
\usepackage{eucal}
\usepackage[all]{xy}
\usepackage{graphicx}
\usepackage{tikz}
\usepackage{tikz-cd}
\usepackage{latexsym,epsfig,verbatim}
\usepackage[margin=1.1in,letterpaper,portrait]{geometry}
%\usepackage[colorinlistoftodos, textsize=tiny]{todonotes}
\usepackage{thmtools}
\usepackage{thm-restate}
\usetikzlibrary{matrix,calc, arrows,decorations.pathmorphing}


\theoremstyle{plain}
  \newtheorem{thm}{Theorem}[section]
  \newtheorem{prop}[thm]{Proposition}
  \newtheorem{lem}[thm]{Lemma}
  \newtheorem{cor}[thm]{Corollary}
  \newtheorem{conjecture}[thm]{Conjecture}
\theoremstyle{definition}
  \newtheorem{defn}[thm]{Definition}
  \newtheorem{rem}[thm]{Remark}
  \newtheorem{eg}[thm]{Example}
  \newtheorem{ex}[thm]{Exercise}
  \newtheorem{question}[thm]{Question}
  \newtheorem{counterexample}[thm]{Counterexample}
  \newtheorem{convention}[thm]{Convention}
\theoremstyle{remark}
  \newtheorem{note}[thm]{Notation}
  \newtheorem*{note*}{Notation}

\numberwithin{equation}{section}

\newtheorem*{conjecture*}{Conjecture}
\newtheorem*{acknowledgement}{ACKNOWLEDGEMENT}

\newcommand\nc{\newcommand}
\nc\on{\operatorname}
\nc\renc{\renewcommand}
\newcommand\ssec{\subsection}
\newcommand\sssec{\subsubsection}
\renewcommand{\iff}{\Leftrightarrow}
\newcommand{\id}{\mathrm{id}}
\newcommand\rk{\operatorname{rk}}
\newcommand\fbn{\mathcal H}
\newcommand \edgequot{\text{edge-quotient bijective }}

\def\Stab{\operatorname{Stab}}
\def\Fix{\operatorname{Fix}}
\def\Aut{\operatorname{Aut}}
\def\op{\operatorname{op}}
\newcommand\im{\text{Im}}
\fi

% etc.
%
% please place your own definitions here and don't use \def but
% \newcommand{}{}
%
% Insert the name of "your journal" with
% \journalname{myjournal}
%
\begin{document}

\title{Peckness of Edge Posets}

\author{David Hemminger      \and
        Aaron Landesman 		 \and
        Zijian Yao
}

%\authorrunning{Short form of author list} % if too long for running head

\institute{D. Hemminger \at
              Duke University
              \email{david.hemminger@duke.edu}           %  \\
%             \emph{Present address:} of F. Author  %  if needed
           \and
           A. Landesman \at
             Harvard University
              \email{aaronlandesman@gmail.com}
           Z. Yao \at
             Brown University
              \email{yaozijian1992@gmail.com}
}

\date{March 16, 2015}
% The correct dates will be entered by the editor


\maketitle

\begin{abstract}
For any graded poset $P$, we define a new graded poset, $\mathcal E(P)$, whose elements are the edges in the Hasse diagram of P. For any group, $G$, acting on the boolean algebra, $B_n$, we conjecture that $\mathcal E(B_n/G)$ is Peck. We prove that the conjecture holds for ``common cover transitive'' actions. We give some infinite families of common cover transitive actions and show that the common cover transitive actions are closed under direct and semidirect products.
\keywords{Peck Posets \and Boolean Algebra \and Group Actions \and Unimodality}

\end{abstract}

%%%%%%%%%%%%%%%%% Introduction %%%%%%%%%%%%%%%%%%
%%%%%%%%%%%%%%%%%%%%%%%%%%%%%%%%%%%%%%%%%%%%%
\section{Introduction}\label{sec:introduction}

\end{document}